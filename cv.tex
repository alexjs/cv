% Formatting "borrowed" from http://www.toofishes.net/blog/latex-resume-follow-up/
% Originally used currvita but changed due to masses of excess whitespace

\documentclass[10pt,a4paper]{article}
\usepackage[a4paper,margin=0.75in]{geometry}
\usepackage[utf8]{inputenc}
\usepackage{mdwlist}
\usepackage[T1]{fontenc}
\usepackage{textcomp}
\usepackage{tgpagella}
\pagestyle{empty}
\setlength{\tabcolsep}{0em}

% indentsection style, used for sections that aren't already in lists
% that need indentation to the level of all text in the document
\newenvironment{indentsection}[1]%
{\begin{list}{}%
    {\setlength{\leftmargin}{#1}}%
    \item[]%
}
{\end{list}}

% opposite of above, bump a section back toward the left margin
\newenvironment{unindentsection}[1]%
{\begin{list}{}%
    {\setlength{\leftmargin}{-0.5#1}}%
    \item[]%
}
{\end{list}}

% format two pieces of text, one left aligned and one right aligned
\newcommand{\headerrow}[2]
{\begin{tabular*}{\linewidth}{l@{\extracolsep{\fill}}r}
    #1 &
    #2 \\
\end{tabular*}}

% and the actual content starts here
\begin{document}

\begin{center}
{\LARGE \textbf{Alex Smith}}


alex@alexsmith.org\ \ \textbullet
\ \ Singapore\ \ \textbullet
\ \ http://alexsmith.org
\\
APAC (Pri) +65 8451 6184 \ \ \textbullet
\ \ EMEA (Sec) +44 7525 909 332
\end{center}

\hrule
\vspace{-0.4em}
\subsection*{Objective}
    My speciality is to drop in to any situation and quickly build technical leadership, regardless of the discipline involved. Thriving in complex environments with a range of unusual and challenging clients, I want to apply this where I can provide significant impact and improvement.
    \\

\hrule
\vspace{-0.4em}
\subsection*{Experience}

\begin{itemize}
    \parskip=0.1em

    \item
        \headerrow
            {\textbf{SwiftServe}}
            {\textbf{Singapore \& Cambridge, UK}}
        \\
        \headerrow
            {\emph{Head of Professional Services}}
            {\emph{October 2013 -- Present}}
    
            SwiftServe is a content delivery technology provider, with principle offices in Singapore and Cambridge, UK. SwiftServe's technology spans both Transparent Caching - for deployment into ISP environments to reduce bandwidth costs and improve user experience; and as Content Delivery Networking - allowing website operators to improve site performance and scalability.

            Joining SwiftServe to build the Professional Services department, primarily focussed on expanding the capabilities of the network to compete in a quickly changing market. As the technical leader of the Singapore HQ, heavily involved in the product management lifecycle, representing the user and commercial requirements from across the customer base in the development and network roadmap.
            
            \begin{itemize*}
                \item Reporting directly to the CEO
                \item Lead the development of the Web Application Firewall project (Layer 7 DDOS protection)
                \item Build Professional Services department from scratch, up to multiple heads and a multi-S\$mm revenue
                \item Post restructure, consolidated deployment team into Professional Services
                \item Extensive technical investigation, resolving high profile repeated faults and improving overall customer experience
            \end{itemize*}

    \item
    \headerrow
        {\textbf{Piksel (formerly KIT digital, ioko)}}
        {\textbf{London, UK}}
    \headerrow
        {\emph{Senior Technical Architect}}
        {\emph{01/2012 -- 10/2013}}
    \headerrow
        {\textbf{ioko}}
        {\textbf{London, UK}}   
    \headerrow
        {\emph{Technical Lead, Senior Engineer, Systems Engineer}}
        {\emph{03/2008 -- 12/2011}}

    Piksel provides video on demand product development, ITIL managed services and technical consultancy/professional services to a range of clients, including Channel 4, BBC, AT\&T, and Disney. Piksel purchased ioko in April 2011 to strengthen their professional services and technical development arm.    

    \begin{itemize*}
        \item Worked as the principle technical authority on several projects, mentoring teams of engineers for the successful delivery of high profile platforms. Involvement ranging from advising C level executives on technical strategy to tracing system calls
        \item Primary technical interface to clients, proced both high and low level designs, with deep involvement in technical implementations; acted as the SME for a range of products (internal and external) and technologies
        \item Key account contact for multiple customers, instrumental in both client retention and account development, through thorough technical assurance
        \item Developed in-house skunkworks-style projects which have been adopted into the product development organization development, which form part of customer production systems
        \item In addition to technical roles, taken on commercial, pre-sales and marketing responsibilities - including exhibiting at IBC - the main conference for broadcasters, and participating in several extensive proposals in both an architectural and commercial role
        \item Knowledge of European, Middle Eastern and South East Asian video markets and working environments
    \end{itemize*}
    \subsection*{\emph{Projects}}
        \begin{itemize}
            \item 
            \headerrow
                {\textbf{Celcom, Axiata Group Berhad}}
                {\textbf{Kuala Lumpur, Malaysia}}
            \\
            \headerrow
                {\emph{Senior Technical Architect}}
                {\emph{2012 - 2013}}
                Video on Demand project, to deliver premium content to existing cellular subscribers. Working primarily as the Platform Architect, specialising in the development of automated deployment systems and platform content delivery strategy.
            \item 
            \headerrow
                {\textbf{Channel 4 Television Corporation}}
                {\textbf{London, UK}}
            \\
            \headerrow
                {\emph{Lead Infrastructure Architect}}
                {\emph{2010 - 2012}}
                Channel 4 is the UK's alternative public broadcaster. A technology innvator, they aggressively target new platforms, with a strategy underpinned by the use of public and private IaaS providers. Working as the primary infrastructure architect on a range of services, and final line of technical escalation for production issues.
                \begin{itemize*}
                    \item Development and implementation of an IaaS Management System across all Amazon EC2 and Private Cloud hosted sites
                    \item Lead Infrastructure Architect:
                    \begin{itemize*}
                        \item 4oD on PS3, iOS, and XBox 360 (winner of the IBC 2012 Innovation in Content Delivery Award)
                        \item Full Platform Re-architecture (to AWS)
                        \item 4Music 2011 Re-launch (http://www.4music.com)
                        \item Scrapbook (http://scrapbook.channel4.com)
                    \end{itemize*}
                \end{itemize*}
        
            \item 
            \headerrow
                {\textbf{FilmFlex (Film4oD, Virgin On Demand, hmvon-demand)}}
                {\textbf{London, UK}}
            \\
            \headerrow
                {\emph{Infrastructure Platform Owner}}
                {\emph{2009 - 2013}}
                FilmFlex is a UK Movies on Demand service. Acting as the internal infrastructure platform owner for this platform within KIT digital, and managing and maintaining the strategic platform architecture. During the tenure with FilmFlex, the platform took on three new affiliates, launched on mobile devices, and had a daily transaction rate increase of several orders of magnitude. FilmFlex was purchased in 2014 by Vubiquity.
            \item 
            \headerrow
                {\textbf{SeeSaw.com / Project Kangaroo}}
                {\textbf{London, UK}}
            \\
            \headerrow
                {\emph{Various}}
                {\emph{2008 - 2010}}
                Project Kangaroo was initially a IPTV venture by the BBC, Channel 4 and ITV, later launched as SeeSaw.com. Worked first in the application support team, transitioned to lead project engineer, and lead the architecture of the data warehousing and reporting components.
            \item 
            \headerrow
                {\textbf{FOXTEL}}
                {\textbf{Sydney, Australia}}
            \\
            \headerrow
                {\emph{Senior Engineer}}
                {\emph{2010}}
                FOXTEL is an Australian premium television company. Originally joining the team to aid the implementation, relocated to Australia to provide technical assurance for meeting aggressive launch dates for the new FOXTEL on Xbox 360 platform.
            \item 
            \headerrow
                {\textbf{BBC Monitoring}}
                {\textbf{UK}}
            \\
            \headerrow
                {\emph{Senior Engineer}}
                {\emph{2009}}
                BBC Monitoring is a commercial arm of the BBC, which provides signal intelligence across various data sources. ioko provided technical consultancy for the infrastructure and application deployments. Challenges included rapid deployment of a large estate, systems automation on Solaris 10, vast signal acquisition, multi-site fully redundancy with data replication (both ZFS and Oracle RDBMS).
        \end{itemize}

    \item
        \headerrow
            {\textbf{Venda, Inc.}}
            {\textbf{London, UK}}
        \\
        \headerrow
            {\emph{Systems Administrator}}
            {\emph{2007 -- 2008}}
    
            Venda, Inc. was the world's largest SaaS e-commerce provider for the world's leading retailers and manufacturers. Their PCI-DSS Tier One compliant platform handled hundreds of transactions per second. Venda was acquired in 2014 by NetSuite Inc.
    \item
        \headerrow
            {\textbf{aql}}
            {\textbf{Leeds, UK}}
        \\
        \headerrow
            {\emph{Systems Support \& Software Development}}
            {\emph{2004 -- 2006 (various)}}
    
            aql are an award winning VoIP, datacentre and hosting company based in the UK.
\end{itemize}


\hrule
\vspace{-0.4em}
\subsection*{Core Technical Skills}

\begin{indentsection}{\parindent}
\hyphenpenalty=1000
\begin{description*}
    \item[Technologies:]
    Web caching, Nginx, Varnish, Apache, RiverBed STM, Puppet, Tomcat, Jetty, MongoDB, Oracle RDBMS, MySQL, Cisco IOS, Juniper JunOS, CheckPoint FW-1, VMware ESX, Solaris 10, Xen, KVM, Private Cloud (Eucalyptus/OpenStack), NetApp, EMC, Nagios, Wowza, GlusterFS
    \item[Languages:]
    Write: Python, \LaTeX, Perl, shell script, SQL; Read: Java, C, C++, Ruby, etc.
\end{description*}
\end{indentsection}


\hrule
\vspace{-0.4em}
\subsection*{Talks}
    \headerrow
        {Test Driven Infrastructure (Internal - Piksel)}
        {June 2012}
    \headerrow
        {MongoDB For an RDBMS DBA (Internal - Piksel)}
        {May 2012}
    \headerrow
        {DevOps in an ITIL Environment (VMware London, internal - Piksel)}
        {Sep 2011 / Jan 2012}

\end{document}
